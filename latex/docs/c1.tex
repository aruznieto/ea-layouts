%%%%%%%%%%%%%%%%%%%%%%%%%%%%%%%%%%%%%%%%%%%%%%%%%%%%%%%%%%%%%%%%%%%%%%%%
% PLANTILLA AC
% Realizado por: C.C. Andrés Ruz Nieto (CGEA-CBE-CTP)
%%%%%%%%%%%%%%%%%%%%%%%%%%%%%%%%%%%%%%%%%%%%%%%%%%%%%%%%%%%%%%%%%%%%%%%%

\chapter{Código de interés} \label{ch:código}

\section{Tablas} \label{sec:tablas}

    \begin{table}[H]
    \small
      \centering
        \extrarowheight = -0.2ex
        \renewcommand{\arraystretch}{1.75}
        \noindent\makebox[\textwidth]{
            \begin{tabular}{|c|c|}
            \hline
            \rowcolor[RGB]{229,229,229} \textbf{Columa1} & \textbf{Columa2} \\
            \hline
            1 & 2 \\
            \hline
            \rowcolor[RGB]{229,229,229} 3 & 4 \\
            \hline
            5 & 6 \\
            \hline
            \end{tabular}
            }
        \caption{Tabla}
      \label{tab:tabla}
    \end{table}

\section{Imágenes} \label{sec:imagenes}

    \begin{figure}[H]
    	\centering
    	\includegraphics[width=0.5\textwidth]{img/img1.png}
    	\caption{Imagen}
    	\label{fig:imagen}
    \end{figure}

\section{Código} \label{sec:codigo}

    \begin{lstlisting}[language=Python, caption=Código]
        import numpy
        import pandas
        
        a = 2
        b = 2
        c = a + b
    \end{lstlisting}
    
\section{Grupos de imágenes}

\begin{figure}[H]
    \centering
    \captionsetup[subfigure]{justification=centering}
    \begin{subfigure}{0.4\textwidth}
        \includegraphics[width=\textwidth]{img/img1.png} 
        \caption{Imagen1}
        \label{fig:imagen1}
    \end{subfigure}
    \begin{subfigure}{0.4\textwidth}
        \includegraphics[width=\textwidth]{img/img2.png}
        \caption{Imagen2}
        \label{fig:imagen2}
    \end{subfigure}
    \caption{Grupo de imágenes}
    \label{fig:grupoimagenes}
\end{figure}

\section{Tabla/imagen al lado de tabla/imagen}

\begin{minipage}[c]{0.5\linewidth}
    \begin{figure}[H]
    	\centering
    	\includegraphics[width=0.5\textwidth]{img/img1.png}
    	\caption{Imagen con tabla}
    	\label{fig:imagentabla}
    \end{figure}
\end{minipage}
%\hspace{0.5cm}
\begin{minipage}[c]{0.5\linewidth}
    \begin{table}[H]
    \small
      \centering
        \extrarowheight = -0.2ex
        \renewcommand{\arraystretch}{1.75}
        \noindent\makebox[\textwidth]{
            \begin{tabular}{|c|c|}
            \hline
            \rowcolor[RGB]{229,229,229} \textbf{Columa1} & \textbf{Columa2} \\
            \hline
            1 & 2 \\
            \hline
            \rowcolor[RGB]{229,229,229} 3 & 4 \\
            \hline
            5 & 6 \\
            \hline
            \end{tabular}
            }
        \caption{Tabla con imagen}
      \label{tab:tablaimagen}
    \end{table}
\end{minipage}